\section{Kernel Machines}
With what we have seen so far about SVMs, we find out that:
\begin{itemize}
    \item \textbf{Hard Margin SVMs} can address linearly separable data only;
    \item \textbf{Soft Margin SVMs} can address linearly separable problems with some degree of misclassification;
    \item \textbf{Non-linearly separable problems} need higher expressive power to be addressed.
\end{itemize}
\textbf{Kernel Machines} are an extension of SVMs that can address non-linearly separable problems, mantaining the same underlying principles of SVMs (large margin and theoretical guarantees).
\\The key idea is to \textit{map} the input data into a higher dimensional space, where it is more likely to be linearly separable and perform a linear separation in that space.
\subsection{Feature Map}
\defib{Feature Map}{
    \[
        \phi: \mathcal{X} \rightarrow \mathcal{H}
    \]
    A feature map $\phi$ is a function that maps the input space $\mathcal{X}$ to a higher dimensional (possibly infinite-dimensional) feature space $\mathcal{H}$.
    \\\\All the examples $\boldsymbol{x}_i$ are replaced by their images $\phi(\boldsymbol{x}_i)$ in the feature space. 
    This should increase the expressive power of the representation, introducing features that are combinations of the original inputs.
}
An example of feature map is the following:
\defib{Polynomial Feature Map}{
    \begin{itemize}
        \item \textbf{Homogeneous Polynomial Feature Map} of degree $2$:
    \begin{equation} \label{eq:homogeneous_polynomial_feature_map}
        \phi\!\left(\begin{pmatrix} x_1 \\ x_2 \end{pmatrix}\right) =
        \begin{pmatrix}
            x_1^2 \\
            x_1 x_2 \\
            x_2^2 \\
        \end{pmatrix}
    \end{equation}
    This feature map takes a 2D input vector $\boldsymbol{x} = (x_1, x_2)$ and maps it to a 3D feature space by including polynomial terms of degree 2.
    \item \textbf{Non-Homogeneous Polynomial Feature Map} of degree $2$:
    \begin{equation} \label{eq:non_homogeneous_polynomial_feature_map}
        \phi\!\left(\begin{pmatrix} x_1 \\ x_2 \end{pmatrix}\right) =
        \begin{pmatrix}
            x_1 \\
            x_2 \\
            x_1^2 \\
            x_1 x_2 \\
            x_2^2
        \end{pmatrix}
    \end{equation}
    \end{itemize}
    This feature map is similar to the previous one but includes a bias term (constant $1$) to account for non-homogeneous polynomials.
}
\subsection{Linear separation in feature space}
Once the data is mapped into the feature space using the feature map $\phi$, we can apply SVM algorithms, replacing $\boldsymbol{x}$ with $\phi(\boldsymbol{x})$:
\begin{equation}
    f(x) = \boldsymbol{w}^T \phi(\boldsymbol{x}) + w_0
\end{equation}
A linear separation in the feature space corresponds to a non-linear separation in the original input space.
\\E.g. using the homogeneous polynomial feature map of degree $2$, the decision function becomes:
\[
    \phi\!\left(\begin{pmatrix} x_1 \\ x_2 \end{pmatrix}\right) = sgn(w_1 x_1^2 + w_2 x_1 x_2 + w_3 x_2^2 + w_0)
\]
The linear separator in the $3D$ feature space corresponds to a ellipse in the original $2D$ input space.
\subsection{Kernel trick}
Computing the feature map $\phi(\boldsymbol{x})$ explicitly can be computationally expensive. In the \textbf{dual formulation} 
of SVMs, the data points appear only in the form of \textbf{dot products} $\langle \phi(\boldsymbol{x}), \phi(\boldsymbol{x}') \rangle$.
\defib{Kernel trick}{
The \textbf{kernel trick} consists in replacing the dot product in the feature space with an equivalent \textbf{kernel function} 
    \begin{equation}
        \langle \phi ( \boldsymbol{x}), \phi (\boldsymbol{x}') \rangle
        = \phi(\boldsymbol{x})^T \phi(\boldsymbol{x}') 
        = K(\boldsymbol{x}, \boldsymbol{x}')
    \end{equation}
    The \textbf{kernel functions} uses examples in the \textbf{original input space} $\mathcal{X}$ to compute the dot product in the \textbf{feature space} $\mathcal{H}$, without explicitly computing the feature map $\phi$.
}
\subsubsection{Examples of kernels}
\paragraph{Homogeneous Polynomial Kernel}:
taking in consideration the equation \ref{eq:homogeneous_polynomial_feature_map}, we can derive the following kernel:
\begin{equation}
    K(\boldsymbol{x}, \boldsymbol{x}') = (\boldsymbol{x}^T \boldsymbol{x}')^d
\end{equation}
Setting $d=2$, we have:
\[
    K(\begin{pmatrix} x_1 \\ x_2 \end{pmatrix}, \begin{pmatrix} x_1' \\ x_2' \end{pmatrix}) = (x_1 x_1' + x_2 x_2')^2
\]
\[
    = (x_1 x_1')^2 + 2 x_1 x_2 x_1' x_2' + (x_2 x_2')^2
\]
\[
= \begin{pmatrix}
    x_1^2 \\
    \sqrt{2} x_1 x_2 \\
    x_2^2
\end{pmatrix}^T
\begin{pmatrix}
    x_1'^2 \\
    \sqrt{2} x_1' x_2' \\
    x_2'^2
\end{pmatrix}
\]
This is equivalent to the dot product in the feature space defined by the homogeneous polynomial feature map of degree $2$, up to a scaling factor $\sqrt{2}$.
\paragraph{Non-Homogeneous Polynomial Kernel}:
taking in consideration the equation \ref{eq:non_homogeneous_polynomial_feature_map}, we can derive the following kernel:
\begin{equation}
    K(\boldsymbol{x}, \boldsymbol{x}') = (\boldsymbol{x}^T \boldsymbol{x}' + 1)^d
\end{equation}
Setting $d=2$, we have:
\[
    K(\begin{pmatrix} x_1 \\ x_2 \end{pmatrix}, \begin{pmatrix} x_1' \\ x_2' \end{pmatrix}) = (x_1 x_1' + x_2 x_2' + 1)^2
\]
\[
    = (x_1 x_1')^2 + 2 x_1 x_2 x_1' x_2' + (x_2 x_2')^2 + 2 x_1 x_1' + 2 x_2 x_2D   + 1
\]
\[= \begin{pmatrix}
    x_1^2 \\
    \sqrt{2} x_1 x_2 \\
    x_2^2 \\
    \sqrt{2} x_1 \\
    \sqrt{2} x_2 \\
    1
\end{pmatrix}^T
\begin{pmatrix}
    x_1'^2 \\
    \sqrt{2} x_1' x_2' \\
    x_2'^2 \\
    \sqrt{2} x_1' \\
    \sqrt{2} x_2' \\
    1
\end{pmatrix}
\]
This is equivalent to the dot product in the feature space defined by the non-homogeneous polynomial feature map of degree $2$, up to a scaling factor $\sqrt{2}$.
\subsubsection{Valid Kernels}
\defib{Valid Kernel}{
    A kernel function $K : \mathcal{X} \times \mathcal{X} \to \mathbb{R}$ is valid if it corresponds to a dot product in some feature space $\mathcal{H}$:
    \begin{equation}
        K(\boldsymbol{x}, \boldsymbol{x}') = \langle \phi(\boldsymbol{x}), \phi(\boldsymbol{x}') \rangle
    \end{equation}
}
The kernel generalizes the concept of similarity between data points in arbitrary feature spaces.
\defib{Gram Matrix}{
    Given examples $\{\boldsymbol{x}_1, \boldsymbol{x}_2, \ldots, \boldsymbol{x}_m\}$, and a kernel function $k$, the \textbf{Gram matrix} $K$ is defined as:
    \begin{equation}
        K_{ij} = k(\boldsymbol{x}_i, \boldsymbol{x}_j) \forall i, j
    \end{equation}
}
\defib{Positive definite matrix}{
    A matrix $M \in \mathbb{R}^{m \times m}$ is positive definite if:
    \begin{equation}
        \sum_{i,j = 1}^m c_i c_j K_{ij} \geq 0 \quad \forall \boldsymbol{c} \in \mathbb{R}^m
    \end{equation}
}
A sufficient and necessary condition for a kernel $K$ to be valid is that the corresponding Gram matrix is positive definite for any choice of examples $\{\boldsymbol{x}_1, \boldsymbol{x}_2, \ldots, \boldsymbol{x}_m\}$.
There are several ways to check if a kernel is valid, e.g.:
\begin{itemize}
    \item Prove its positive definiteness;
    \item Find out a corresponding feature map $\phi$ (as we did before);
    \item Use kernel combination properties (e.g. sum, product yield valid kernels).
\end{itemize}
\subsection{Basic Kernels}
Basic kernels can be useful as building blocks for more complex kernels. Some common examples include:
\begin{itemize}
    \item \textbf{Linear Kernel}:
    \begin{equation}
        K(\boldsymbol{x}, \boldsymbol{x}') = \boldsymbol{x}^T \boldsymbol{x}'
    \end{equation}
    \item \textbf{Polynomial Kernel}:
    \begin{equation}
        K_{d,c}(\boldsymbol{x}, \boldsymbol{x}') = (\boldsymbol{x}^T \boldsymbol{x}' + c)^d
    \end{equation}
    \item \textbf{Gaussian Kernel}:
    \begin{equation}
        k_\sigma(\boldsymbol{x}, \boldsymbol{x}') = \exp\left(-\frac{\|\boldsymbol{x} - \boldsymbol{x}'\|^2}{2\sigma^2}\right)
        = \exp\left(-\frac{\boldsymbol{x}^T \boldsymbol{x} + \boldsymbol{x}'^T \boldsymbol{x}' - 2 \boldsymbol{x}^T \boldsymbol{x}'}{2\sigma^2}\right) 
    \end{equation}
    The kernel depends on a parameter $\sigma$ that controls the width of the Gaussian function. The smaller the $\sigma$, the more localized the influence of each training example.
\end{itemize}
The Gaussian kernel is an example of \textbf{universal kernel}, which means that it can approximate any continuous function on a compact domain given enough data.
\subsection{Kernel combinations}
Basic kernels can be \textbf{combined} using various operations to create more complex kernels from simpler ones. 
Correctly using combination operators guarantees that the resulting kernel is still positive definite.
\\We will now analyze some common kernel combination methods.
\subsubsection{Kernel Sum}
The sum of two kernels of two kernels corresponds to the concatenation of their feature spaces:
\begin{equation}
    K(\boldsymbol{x}, \boldsymbol{x}') = K_1(\boldsymbol{x}, \boldsymbol{x}') + K_2(\boldsymbol{x}, \boldsymbol{x}')
\end{equation}
Applying the definition of kernel:
\[
= \phi_1(\boldsymbol{x})^T \phi_1(\boldsymbol{x}') + \phi_2(\boldsymbol{x})^T \phi_2(\boldsymbol{x}')
\]
\[
= (\phi_1(\boldsymbol{x}) + \phi_2(\boldsymbol{x}))^T (\phi_1(\boldsymbol{x}') + \phi_2(\boldsymbol{x}'))
\]
\subsubsection{Kernel Product}
The product of two kernels corresponds to the \textbf{cartesian product} of their features:
\begin{equation}
    (k_1 \times k_2)(\boldsymbol{x}, \boldsymbol{x}') = K_1(\boldsymbol{x}, \boldsymbol{x}') K_2(\boldsymbol{x}, \boldsymbol{x}')
\end{equation}
\[
= \sum_{i = 1}^n \phi_{1i}(\boldsymbol{x})\phi_{1i}(\boldsymbol{x}') \sum_{j = 1}^m \phi_{2j}(\boldsymbol{x})\phi_{2j}(\boldsymbol{x}')
\]
\[
= \sum_{i = 1}^n \sum_{j = 1}^m (\phi_{1i}(\boldsymbol{x}) \phi_{2j}(\boldsymbol{x})) (\phi_{1i}(\boldsymbol{x}') \phi_{2j}(\boldsymbol{x}'))
\]
\[
= \sum_{k = 1}^{nm} \psi_{12k}(\boldsymbol{x}) \psi_{12k}(\boldsymbol{x}') = \psi_{12}(\boldsymbol{x})^T \psi_{12}(\boldsymbol{x}')
\]
where $\psi_{12}(\boldsymbol{x}) = \phi_{1}(\boldsymbol{x}) \times \phi_{2}(\boldsymbol{x})$ is the cartesian product of the two feature maps.
The product can also be between kernel in different spaces.
\subsubsection{Linear combination of kernels}
A kernel can be rescaled by a positive constant:
\[ k_{\beta}(\boldsymbol{x}, \boldsymbol{x}') = \beta k(\boldsymbol{x}, \boldsymbol{x}')\]
We can define a linear combination of kernels as:
\[k_{sum}(\boldsymbol{x}, \boldsymbol{x}') = \sum_{k=1}^K \beta_k k_k(\boldsymbol{x}, \boldsymbol{x}') \]
Since understanding what kernel works best for a given problem can be challenging, combining multiple kernels allows us to leverage the strengths of each individual kernel.
With \textbf{kernel learning} the weights $\beta_k$ of the combination can be learned from data, allowing the model to adaptively combine multiple kernels.
\subsubsection{Kernel normalization}
Kernel values can be influenced by the dimension of objects. I.E., in text classification, longer documents tend to have higher dot products simply because they contain more words.
\\To mitigate this effect, we can normalize the kernel:
\begin{equation}
    \hat{K}(\boldsymbol{x}, \boldsymbol{x}') = \frac{K(\boldsymbol{x}, \boldsymbol{x}')}{\sqrt{K(\boldsymbol{x}, \boldsymbol{x}) K(\boldsymbol{x}', \boldsymbol{x}')}}
\end{equation}
This is a \textbf{cosine normalization}, which takes into account the angle between the feature vectors rather than their magnitudes.
\subsection{Kernel on Graphs}
Kernels on graphs heavily rely on the Weistfeiler-Lehman (WL) test for graph isomorphism.
\subsubsection{Graph Isomorphism}
To understand this algorithm, we need to define the concept of \textbf{graph isomorphism}.
\defib{Graph Isomorphism}{
    Two graphs $G = (V, E)$ and $G' = (V', E')$ are isomorphic if there exists a bijection $f: V \to V'$ such that:
    \begin{equation}
        (u, v) \in E \iff (f(u), f(v)) \in E'
    \end{equation}
}
In a nutshell, two graphs are isomorphic if they have the same structure, but possibly different node labels.
\subsubsection{Weisfeiler-Lehman Test}
If isomorphism is easy to check on vectorial data, it is much more challenging on graph data. This is in fact an NP-hard problem.
\\The WL test is an approximate algorithm that can efficiently determine if two graphs are non-isomorphic, but it may fail to distinguish some non-isomorphic graphs (false positives).
\paragraph{Algorithm:}
We are now going to present the WL algorithm on two graphs with same node labels:
\begin{itemize}
    \item Given two graphs $G = (V, E)$ and $G' = (V', E')$ with $|V| = |V'| = n$;
    \item Let $L(G) = {l(v)| v \in V} $ be the set of node labels in $G$. $l(v)$ is a function returning the label of node $v$; 
    \item Let $L(G') = L(G)$;
    \item Let $label(s)$ be a function assigning a unique label to a string.
\end{itemize}
The algorithm is defined as follows:
\begin{algorithm}[H]
    \caption{Weisfeiler-Lehman Test}
    \begin{algorithmic}
        \State Set $l_0(v) \leftarrow l(v), \forall v \in V$
        \For{$i = 1$ to $n-1$}
            \For{each node $v \in G$}
                \State Let $M_i(v) = \{l_{i-1}(u) | u \in \text{neigh}(v)\}$
                \State Concatenate the sorted labels of $M_i(v)$ into $s_i(v)$
                \State Let $l_i(v) = label(l_{i-1}(v) \oplus s_i(v))$
            \EndFor
            \State If $L_i(G) \neq L_i(G')$, return \textbf{not isomorphic}
        \EndFor
        \State Return \textbf{possibly isomorphic}
    \end{algorithmic}
\end{algorithm}
Where:
\begin{itemize}
    \item $M_i(v)$: the multiset of labels of the neighbors of node $v$ at iteration $i$;
    \item $l_i(v)$: the new label assigned to node $v$ at iteration $i$. It is computed by assigning a unique label 
    to the concatenation of the previous label $l_{i-1}(v)$ and the sorted labels of its neighbors $s_i(v)$.
\end{itemize}
\subsubsection{Weisfeiler-Lehman Kernel}
\begin{itemize}
    \item Let ${G_1, G_2, \ldots, G_h} = {(V, E, l_0), (V, E, l_1), \ldots, (V, E, l_h)}$ be a set of resulting graphs after applying $h$ iterations of the WL algorithm on graph $G$.
    \item Let $k: G \times G \to \mathbb{R}$ be any kernel on graphs;
\end{itemize}
The \textbf{Weisfeiler-Lehman kernel} between two graphs $G$ and $G'$ is defined as:
\begin{equation}
    K_{WL}^{h}(G, G') = \sum_{i=0}^h k(G_i, G_i')
\end{equation}
