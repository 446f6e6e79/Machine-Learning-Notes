% Pacchetti essenziali
\usepackage{graphicx} % Required for inserting images
\usepackage{comment}
\usepackage{amssymb}
\usepackage{amsmath}
\usepackage{amsthm}
\usepackage{graphicx}
\usepackage{float}
\usepackage{comment}
\usepackage{caption}
\usepackage{multirow}
\usepackage{listings}
\usepackage{xcolor}
\usepackage[hidelinks]{hyperref}
\usepackage{enumitem}
\usepackage{titlesec}
\usepackage{algorithm}
\usepackage{algpseudocode}
\usepackage{tcolorbox} 
\usepackage{fancyhdr} % Aggiunto per intestazioni personalizzate
\usepackage{mdframed} 
\usepackage{tikz}
\usetikzlibrary{arrows.meta, calc, positioning, backgrounds}
\usepackage{pgfplots}
\usetikzlibrary{arrows.meta, calc}

% Command to get the current month name
\newcommand{\MonthName}{
  \ifcase\month\or January\or February\or March\or April\or May\or June\or July\or August\or September\or October\or November\or December\fi}
  
% Configurazione di codeblocks con il pacchetto listings
\lstset{
  basicstyle=\ttfamily\small,
  keywordstyle=\color{blue},
  commentstyle=\color{green!50!black},
  stringstyle=\color{red},
  showstringspaces=false,
  breaklines=true,
  frame=single, % Aggiunge un bordo attorno al codice
  numbers=left, % Numeri di riga a sinistra
  numberstyle=\tiny\color{gray}, % Stile dei numeri di riga
  stepnumber=1, % Numero di righe da saltare tra i numeri
  numbersep=-5pt, % Distanza tra i numeri di riga e il codice
  tabsize=2, % Dimensione del tab
  captionpos=b, % Posizione della didascalia
}

% Configurazione dell'intestazione
\pagestyle{fancy}
\fancyhf{} % Pulisce le intestazioni e i piè di pagina
\fancyhead[L]{\leftmark} % Sezione corrente a sinistra
\fancyhead[R]{\thepage} % Numero di pagina a destra

% Comandi personalizzati per definizioni
\newcommand{\definition}[2]{%
  \paragraph{#1:} #2%
}

% Alternativa con enfasi sul termine
\newcommand{\defterm}[2]{%
  \paragraph{\textbf{#1}:} #2%
}

% Ambiente per liste di definizioni
\newenvironment{definitions}{%
  \begin{description}[font=\normalfont\bfseries\large, leftmargin=0pt, labelindent=0pt]
}{%
  \end{description}
}

% Definizione con box colorato (opzionale)
\newtcolorbox{defbox}[1]{
  colback=blue!5!white,
  colframe=blue!75!black,
  title=#1,
  fonttitle=\bfseries
}

% Defines a command for framed definitions with a blue background and a title
\newcommand{\defi}[1]{\begin{mdframed} [nobreak=true,hidealllines=false,linecolor=blue!40,linewidth=2pt,backgroundcolor=blue!5,]{#1} \end{mdframed}}
\newcommand{\defib}[2]{\begin{mdframed} [hidealllines=false,linecolor=blue!40,linewidth=2pt,backgroundcolor=blue!5,] \textbf{{#1}} \vspace{2mm} \\ {#2} \end{mdframed}}

% Make proof keyword bold
\makeatletter
\def\proofname{\bfseries Proof}
\makeatother
% PGFPlots settings
\pgfplotsset{compat=1.18}

%tikz styles for Bayesian Networks graphs
\tikzset{
  bn-node/.style={circle, draw=red, thick, minimum size=18pt},
  bn-arrow/.style={->, red, thick},
  % D-separation styles
  dse-node-empty/.style={circle, draw=red, fill=orange!10, minimum size=6mm, inner sep=0pt},
  dse-node-filled/.style={circle, draw=red, fill=blue!60, minimum size=6mm, inner sep=0pt},
  dse-indep-bg/.style={ellipse, fill=yellow!25, draw=black, thick, minimum width=7.5cm, minimum height=2.8cm},
  dse-dep-bg/.style={ellipse, fill=green!30, draw=black, thick, minimum width=7.5cm, minimum height=2.8cm},
  dse-small-indep/.style={ellipse, fill=yellow!25, draw=black, thick, minimum width=2.8cm, minimum height=2.2cm},
  dse-small-dep/.style={ellipse, fill=green!30, draw=black, thick, minimum width=2.8cm, minimum height=2.2cm},
  dse-boundary/.style={ellipse, draw=black, dashed, minimum width=3cm, minimum height=6.5cm}
}